\documentclass[11pt,a4paper]{article}
\usepackage{acl2015}
\usepackage{times}
\usepackage{url}
\usepackage{latexsym}

\title{Asignatura Text Mining en Social Media. Master Big Data}

\author{Lorena Ponce Ruiz, Sergio Langarita Benítez, Luis Miguel Bartolin Arnau \\
  {\tt lorena.ponru@gmail.com} \\
  {\tt sergiolangaritabenitez@gmail.com} \\
  {\tt luismibartolinarnau@gmail.com} \\}

\date{}

\begin{document}
\maketitle
\begin{abstract}
Dado un dataset de tweets de diferentes usuarios de twitter. Predecir si un autor es irónico o no. Para ello se han planteado diferentes modelos de predicción, y tratamientos de los datos.
%Aqu\'i un resumen de unas 250 palabras de c\'omo se ha aproximado la tarea
\end{abstract}


\section{Introducción}


Dado un dataset con un conjunto de tweets 8400 se ha separado el dataset en dos partes: training y test. Se ha entrenado diferentes modelos con los tweets de la parte training y para validar el modelo se ha buscado predecir si los tweets de la parte de test son irónicos o no.





 


%Breve introducci\'on al problema de Author Profiling y concretamente al presentado en clase. En este apartado el alumno deber\'a resumir en qu\'e consiste el problema y ponerlo en perspectiva para que el lector comprenda los siguientes apartados.


\section{Dataset}

Se ha proporcionado una carpeta 421 archivos, solo el archivo 'truth.txt' tiene un formato csv. Donde hay dos columnas, la primera columna representa el autor y en la segunda columna etiqueta si es un autor irónico o no. Respecto al resto de los archivos, cada archivo contiene 20 tweets de un autor, en formato xml. El nombre de estos archivos es el nombre del autor. Previamente, se habían anonimizado los nombres de los archivos para ocultar sus autores, y dentro de los archivos se habían anonimizado los hashtags, las menciones a otros usuarios y las urls.


%Estad\'isticas del dataset que el alumno considere importantes. En clase se han visto las estad\'isticas b\'asicas del dataset y se ha explorado para obtener caracter\'isticas m\'as avanzadas. En este apartado el alumno tiene total libertad para exponer las tablas o gr\'aficas que considere apropiadas para describir el dataset, tanto desde un punto de vista ling¨u\'istico como de big data. 


\section{Propuesta del alumno}


Se ha creado un Bag of Words respecto al training y se ha realizado un árbol de decisión y random forest. Y con este Bag of Words se ha generado otro dataframe dibujando un Árbol de decisión y seleccionando las variables más relevantes. Y también se ha producido un tercer dataframe con una matriz de correlación de las variables entre sí y se han seleccionado las variables con más y con menos correlación respecto a la columna de resultado. Para estos tres dataframes se han propuesto el modelo lineal generalizado (glm) y regresión lineal (lm). Otro tratamiento que se ha realizado a los datos es la creación de una matriz Tf-idf (Term frequency - Inverse document frequency) y aplicar sobre esta matriz una red neuronal.








%Descripci\'on de la propuesta. Qu\'e caracter\'isticas se han utilizado y cu\'al ha sido la hip\'otesis para elegirlas. En clase se ha visto la construcci\'on de una baseline basada en bolsa de palabras. En este apartado el alumno expondr\'a las mejoras propuestas.

\section{Resultados experimentales}

Después de evaluar los diferentes modelos, se ha llegado a la conclusión que el random forest es el modelo que mejores resultados puede dar, así que se ha experimentado con diferente número de palabras en el vocabulario. Se ha probado hasta con, 15000 palabras, pero bajaban los resultados. Con 1000 palabras se ha llegado a 0.95 de accuracy y 0.9 de kappa, sobre el test.


%Presentaci\'on de los resultados y an\'alisis de los mismos. La presentaci\'on de resultados y su an\'alisis implica mostrar en qu\'e contribuye la propuesta realizada, es decir, ¿son mejores los resultados?, ¿se procesan m\'as r\'apidos los datos?, ¿se aportan nuevas explicaciones conceptuales al problema?

\section{Conclusiones y trabajo futuro}

El mejor resultado obtenido ha sido mediante la creación de un Bag of Words y la aplicación de un random forest. Con respecto al trabajo futuro se puede centrar en la manipulación del dataset, dando más peso a elementos como los emoticonos, hastags, las menciones a otros usuarios y seleccionando diferentes palabras claves.


%Breve presentaci\'on de las conclusiones sobre  el trabajo realizado e ideas de futuro para mejorar los resultados.


%\begin{thebibliography}{}

%\bibitem[\protect\citename{Aho and Ullman}1972]{Aho:72}
%Alfred~V. Aho and Jeffrey~D. Ullman.
%\newblock 1972.
%\newblock {\em The Theory of Parsing, Translation and Compiling}, volume~1.
%\newblock Prentice-{Hall}, Englewood Cliffs, NJ.

%\end{thebibliography}

\end{document}
